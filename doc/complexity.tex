\documentclass[a4paper]{ctexart}

\usepackage[left=2.50cm, right=2.50cm, top=2.50cm, bottom=2.50cm]{geometry}

\usepackage{fontspec}

\usepackage{qcircuit}

\usepackage[backref, colorlinks,
    linkcolor=black,
    citecolor=blue,
    urlcolor=magenta
]{hyperref}

\usepackage{amsmath, amsfonts, amssymb, amsthm, extarrows, mathdots}

\numberwithin{figure}{section}

\usepackage{enumerate}

\usepackage{xcolor}

\usepackage{graphicx}

\usepackage{subfigure}

\usepackage{url}

\usepackage{bm}

\usepackage{multirow}

\usepackage{booktabs}

\usepackage{epstopdf}

\usepackage{epsfig}

\usepackage{listings}

\usepackage{longtable}

\usepackage{supertabular}

\usepackage{algorithm}

\usepackage{algorithmic}

\usepackage{changepage}

\lstset{
	basicstyle=\ttfamily,	% 基本样式
		keywordstyle=\color{blue}, % 关键词样式
		commentstyle=\color{gray!50!black!50},   	% 注释样式
		stringstyle=\rmfamily\slshape\color{red}, 	% 字符串样式
	backgroundcolor=\color{gray!5},     % 代码块背景颜色
	frame=leftline,						% 代码框形状
	framerule=12pt,%
		rulecolor=\color{gray!90},      % 代码框颜色
	numbers=left,				% 左侧显示行号往左靠, 还可以为right ,或none,即不加行号
		numberstyle=\footnotesize\itshape,	% 行号的样式
		firstnumber=1,
		stepnumber=1,                  	% 若设置为2,则显示行号为1,3,5
		numbersep=7pt,               	% 行号与代码之间的间距
	aboveskip=.25em, 			% 代码块边框
	showspaces=false,               	% 显示添加特定下划线的空格
	showstringspaces=false,         	% 不显示代码字符串中间的空格标记
	keepspaces=true, 					
	showtabs=false,                 	% 在字符串中显示制表符
	tabsize=2,                     		% 默认缩进2个字符
	captionpos=b,                   	% 将标题位置设置为底部
	flexiblecolumns=true, 			%
	breaklines=true,                	% 设置自动断行
	breakatwhitespace=false,        	% 设置自动中断是否只发生在空格处
	breakautoindent=true,			%
	breakindent=1em, 			%
	title=\lstname,				%
	escapeinside=``,  			% 在``里显示中文
	xleftmargin=1em,  xrightmargin=1em,     % 设定listing左右的空白
	aboveskip=1ex, belowskip=1ex,
	framextopmargin=1pt, framexbottommargin=1pt,
        abovecaptionskip=-2pt,belowcaptionskip=3pt,
	% 设定中文冲突,断行,列模式,数学环境输入,listing数字的样式
	extendedchars=false, columns=flexible, mathescape=true,
	texcl=true,
	fontadjust
}

{
    \theoremstyle{definition}
    \newtheorem{axiom}{\indent 公理}
    \newtheorem{theorem}{\indent 定理}
    \newtheorem{lemma}[theorem]{\indent 引理}
    \newtheorem{proposition}[theorem]{\indent 命题}
    \newtheorem{corollary}[theorem]{\indent 推论}
    \newtheorem{definition}[theorem]{\indent 定义}
    \newtheorem*{solution}{\indent 解}
    \newtheorem{example}{\indent 例}[section]\theoremstyle{definition}
    \newtheorem*{axiom*}{\indent 公理}
    \newtheorem*{theorem*}{\indent 定理}
    \newtheorem*{lemma*}{\indent 引理}
    \newtheorem*{proposition*}{\indent 命题}
    \newtheorem*{corollary*}{\indent 推论}
    \newtheorem*{definition*}{\indent 定义}
    \newtheorem*{example*}{\indent 例}
    \renewcommand{\proofname}{\indent\bf 证明}
}

\renewcommand{\proofname}{\indent\bf 证明}
\newcommand{\bra}[1]{\langle#1|}
\newcommand{\ket}[1]{|#1\rangle}
\newcommand{\inner}[2]{\langle#1|#2\rangle}
\newcommand{\tensor}{\otimes}
\newcommand{\xor}{\oplus}

\newcommand*{\dif}{\mathop{}\!\mathrm{d}}

\setmainfont{Times New Roman}

\usepackage{xpatch}
\makeatletter
\xpatchcmd{\@thm}{\thm@headpunct{.}}{\thm@headpunct{}}{}{}
\makeatother


{
    \theoremstyle{plain}
    \newtheorem*{think}{\indent 思考}
    \newtheorem*{note}{\indent 注}
}

\def\equationautorefname{式}
\def\footnoteautorefname{脚注}
\def\itemautorefname{项}
\def\figureautorefname{图}
\def\tableautorefname{表}
\def\partautorefname{篇}
\def\appendixautorefname{附录}
\def\chapterautorefname{章}
\def\sectionautorefname{节}
\def\subsectionautorefname{小节}
\def\subsubsectionautorefname{小节}
\def\paragraphautorefname{段落}
\def\subparagraphautorefname{子段落}
\def\FancyVerbLineautorefname{行}
\def\theoremautorefname{定理}


\title{斜堆复杂度分析报告}
\author{潘屹 \\ {\small 上海交通大学 ACM 班 (电院 2231), 522031910741}}
\date{}

\begin{document}
\maketitle

\begin{abstract}
    本文对斜堆的时间复杂度进行了分析,参考 D.D. Sleator 与 R.E. Tarjan 的势能分析法\cite{sleator1986self},证明其归并操作的均摊时间复杂度为 \begin{align*}
        T(n) = O(\log n).
    \end{align*}
    进一步地,参考 B. Schoenmakers 的方式\cite{kaldewaij1991derivation}\cite{schoenmakers1997tight},能得到该操作在最坏情况下的均摊时间 \begin{align*}
        T(n) \le \log_\varPhi n,\quad\varPhi=\frac{1+\sqrt{5}}{2}.
    \end{align*}
\end{abstract}

\section{原理简析}

斜堆是左偏树的一个变种,由 D.D. Sleator 与 R.E. Tarjan 在 1986 年提出。其与左偏树的主要区别在于不保证左儿子高度大于等于右儿子,即在 merge 操作中,不进行左右儿子高度的判断;而是每一次将右儿子与新堆归并后,都进行左右儿子的交换操作。归并操作的 C++ 实现代码如下:

\begin{lstlisting}[language=C++]
tnode* merge(tnode *x, tnode *y) {
    if (x == nullptr)
        return y;
    if (y == nullptr)
        return x;
    if (!Compare(x->data, y->data))
        std::swap(x, y);
    merge(x->right, y);
    std::swap(x->left, x->right);
    return x;
}
\end{lstlisting}

通过上述操作,可以对斜堆的原理进行感性理解,即,通过每次右子树归并后交换的方式,尽量维持整棵二叉树的左右子树深度稳定,进行自我调整(self-adjusting),从而使得每一次归并操作的复杂度均摊在 $O(\log n)$\footnote{本文中若无特别说明,$\log$ 表示以 $2$ 为底数的对数。} 。下面的分析将对这一点进行严格证明。

\section{势能分析法}
在 D.D. Sleator 与 R.E. Tarjan 的文章中,使用了势能(Potential)分析的方式进行斜堆均摊复杂度的上界分析。势能分析是一种对数据结构进行均摊复杂度分析的常见方式,对于一个数据结构 $S$(例如,本例中的斜堆),我们可以定义其势能为 $\Phi(S)$,这是一个会随数据结构的形态而改变的值。实际上,可以进行感性理解:当一个数据结构的复杂程度(例如最大深度)增大时,其势能\footnote{Here \textit{potential} refers to its potential to increase later time cost.}也随之增大。因此,定义一个操作 $i$ 的摊还时间消耗 \begin{align*}
    a_i = t_i + (\Phi_i - \Phi_{i-1}).
\end{align*}
于是在一个操作序列 $(op_i)_{i=1}^m$ 中,总时间消耗为 \begin{align*}
    \sum_{i=1}^mt_i=\sum_{i=1}^ma_i-\Delta\Phi.
\end{align*}
通常 $\Phi_i$ 是一个初值为 $0$ 而保持非负的值,这样一来,$\sum a_i$ 就是 $\sum t_i$ 的一个上界。可以看出,势能分析法的关键在于选择适当的势能函数 $\Phi$。下面通过势能分析法对斜堆的复杂度进行分析。

\section{归并操作的复杂度上界}

考虑到斜堆的插入、删除操作的本质实际上都是进行归并操作,所以在分析斜堆的时间复杂度时,只需要对归并操作进行分析即可。

\begin{definition}[重节点]
    对于一个非叶子节点 $u$,其子节点 $v$ 为 $u$ 的\textbf{重儿子}当且仅当 $v$ 的子树大小大于 $u$ 子树大小的一半,并称 $u$ 为树上的\textbf{重节点},反之则为\textbf{轻节点}。
\end{definition}

关于轻重节点有一些显然的推论:一个节点至多只有一个重儿子;一个轻儿子的子树大小不超过其父节点子树大小的一半,从而可以导出引理。

\begin{lemma}
    对于一个非叶子节点 $u$ 与其子树中的一个节点 $v$,记它们的子树大小分别为 $w_u,w_v$,设从 $u$ 到 $v$ 的路径中轻节点个数为 $k$($u$ 不计入),则有 \begin{align*}
        k\le\log\frac{w_u}{w_v}.
    \end{align*}
\end{lemma}
\begin{proof}
    设这条路径的轻节点为 $(x_i)_{i=1}^k$,按从 $u$ 到 $v$ 的路径顺序,并记它们的子树大小为 $w_{x_i}$,则显然有 \begin{align*}
        w_{x_{i+1}}<\frac{w_{x_i}}{2}\ (i\le 1)\quad\text{且}\quad w_{x_1}<\frac{w_u}{2}, w_{v}\le w_{x_k}.
    \end{align*}
    于是 \begin{align*}
        w_{v}\le w_{u}2^{-k}.   
    \end{align*}
\end{proof}
\begin{corollary}
    在节点数为 $n$ 的二叉树的任意一条从顶向下的路径中,共有不超过 $\lfloor \log n\rfloor$ 个轻节点。
\end{corollary}

上面的内容是关于二叉树中轻重节点的基本定义,据此展开斜堆复杂度的分析,首先是最关键的势能函数定义:
\begin{definition}
    对于一个斜堆 $S$,定义其势能函数 $\phi(S)$ 为其所有同时为重儿子与右儿子的节点个数。
\end{definition}
随后通过选定的势能函数进行摊还时间估计:
\begin{theorem}
    对于归并后节点数为 $n$ 的斜堆,其单次归并过程 $i$ 的摊还时间满足 \begin{align*}
        a_i=O(\log n).
    \end{align*}
\end{theorem}
\begin{proof}
    设归并前的两个斜堆 $h_1,h_2$ 的总结点数为 $n_1,n_2$,在其右路径上的重节点个数为 $k_1,k_2$。在一次归并过程中,时间消耗可用其右路径上的节点总数来衡量,即轻儿子与重儿子的个数之和,依引理与相关节点的定义 \begin{align*}
        t_i\le\log n_1+k_1+\log n_2+k_1+k_2\le 2\log n+k_1+k_2.
    \end{align*}
    
    再考虑势能变化,显然,在一次归并操作中,右路径上所有的重节点必然与左子树发生交换,进而转变为轻节点,带来 $-(k_1+k_2)$ 的势能变化;对于轻节点转化为重节点的情况,考虑最坏(势能增加量最大)的情况,即所有路径上的轻节点都转化为重节点,此时势能变化量为 \begin{align*}
        \Phi_i-\Phi_{i-1}\le \log n-k_1-k_2.
    \end{align*}
    于是有 \begin{align*}
        a_i=t_i+\Phi_i-\Phi_{i-1}\le 3\log n=O(\log n).
    \end{align*}
\end{proof}

这样就证明了,斜堆的归并、插入、删除操作的复杂度均为 $O(\log n)$,查询最值的操作为 $O(1)$。

\section{精确时间}

B. Schoenmakers 对斜堆归并操作的时间复杂度进行了更加精确的分析,其中的数学思想较为深刻;下面的内容参考其中方法,得到该操作在最坏情况下的均摊时间。

笔者还没看懂,下次再补。

\bibliographystyle{unsrt}
\bibliography{reference}

\end{document}